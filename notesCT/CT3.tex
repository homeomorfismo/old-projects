\chapter[Lenguajes no contextuales]{Lenguajes no contextuales (o libres de contexto)}

\section[Introducci\'on a las gram\'aticas]{Introducci\'on a las gram\'aticas libres de contexto}

La idea de una gram\'atica es la de ser una colecci\'on de \emph{reglas de substituci\'on}, llamadas \emph{producciones.}
Cada regla cosnta de un s\'imbolo y una cadena separada por una flecha.
El s\'imbolo es llamado \emph{variable}, y las cadenas de s\'imbolos y variables son llamdos \emph{terminales.}
Se designa una \emph{variable inicial,} usualmente es la primera que se expresa en una gram\'atica.
Veamos un ejemplo de una gram\'atica \(G_1.\)

\begin{ejemplo}
    La gram\'atica \(G_1\) esta dada por 
    \[\begin{array}{r l}
        A \to & 0\ A\ 1 \\
        A \to & B \\
        B \to & \#
    \end{array}\]
\end{ejemplo}

Uno puede describir un lenguaje aceptado por la gram\'atica mediante las siguientes reglas:
\begin{enumerate}
    \item Anotar la variable inicial;
    \item Hallar/seleccionar una variable que este escrita, y reemplazarla segun una regla de la gram\'atica;
    \item Repetir hasta que no queden variables escritas. 
\end{enumerate}
A esto se le llama una \emph{derivaci\'on}.
Por ejemplo consideremos la palabra \(000\# 111.\)
Su derivaci\'on corresponde \[A \Rightarrow 0A1 \Rightarrow 00A11 \Rightarrow 000A111 \Rightarrow 000B111 \Rightarrow 000\# 111\]

\section{Gram\'aticas libres de contexto}

Definimos formalmente  una gram\'atica libre de contexto.

\begin{definicion}
    Una gram\'atica libre de contexto es una cu\'adrupla \((V,\Sigma,R,S)\), donde
    \begin{enumerate}
        \item \(V\) es conjunto finito, llamado \emph{variables.}
        \item \(\Sigma\) es conjunto finito disjunto de \(V\), llamado \emph{terminales;}
        \item \(R\) es un conjunto finito de \emph{reglas,} compuestas de varaibles, y una cadena de variables y terminales, y
        \item \(S \in V\) es la variable inicial.
    \end{enumerate}
\end{definicion}

Para \(u,v,w\) cadenas de variables y terminales, y \(A \to w\) una regla, diremos que \(uAv\) \emph{conlleva a } \(uwv\), y lo denotamos por \[uAv \Rightarrow uwv.\]
Anotamos \(u \Rrightarrow v \) si bien \(u=v\) o existen \(\{u_i\}_{i=0}^n\) con \(u=u_0,\) \(v=u_n\) y \(u_i \Rightarrow u_{i+1}\) para todo \(i \in [n-1].\)

Podemos ver que \(A\Rrightarrow 000\# 111\), y para este efecto usaremos un \emph{\'arbol de an\'alisis} (parse tree).
V\'ease la figura \ref{CT3-1}.

\begin{figure}[h!b]
    \centering
    \includegraphics[scale=1]{CT3-1}
    \caption{ Representaci\'on del \'arbol de an\'alisis de la gram\'atica \(G1\) }\label{CT3-1}
\end{figure}

Naturalmente, como ya hemos hecho antes, el lenguaje de las palabras derivadas de la gram\'atica \(G\) se denota por \(L(G)\).

\subsection{Ambiguedad}

Cuando una gram\'atica genera la misma palabra de distintas maneras, decimos que es \emph{ambigua.}
Formalizamos esto introduciendo una modificaci\'on a las reglas.

Para \(u,v,w\) cadenas de variables y terminales, anotamos \(u\to v | w \), si bien \(u\to v\), bien \(u \to w.\)
Una derivaci\'on \emph{izquierda} es aquella que en cada derivaci\'on, la variable m\'as a la izquierda es la sustituida.

\begin{definicion}
    Una palabra \(w\) es derivada \emph{ambiguamente} en una gram\'atica de contexto libre \(G\) si tiene m\'as de una derivaci\'on izquierda. La gram\'atica \(G\) es ambigua si genera una palabra ambiguamente.
\end{definicion}

\begin{obs}
    Notar que, pese a que podemos sustituir eventualmente una gram\'atica ambigua por una mejor, existen lenguajes \emph{inherentemente ambiguos.}
    Un ejemplo de esto es el lenguaje sobre \(\Sigma=\{0,1,2\}\) dado por el conjunto \(\{0^i1^j2^k\ :\ i=j \text{ o bien } j=k\}\).
\end{obs}

\section{Forma normal de Chomsky}

Es \'util trabajar con gram\'aticas relativamente sencillas de manejar. Una de ellas es la forma normal de Chomsky.

\begin{definicion}
    Decimos que una gram\'atica libre de contexto esta en su \emph{forma normal de Chomsky} si cada regla es de la forma \(A\to BC\) o \(A\to a\), donde \(a\) es cualquier terminal y \(A, B, C\) son variables, con \(B,C\) distintas de la variable inicial. En particular se permite la regla \(S\to\epsilon\) con \(S\) variable inicial.
\end{definicion}

\begin{teorema}
    Toda gram\'atica libre de contexto es generada por una gram\'atica libre de contexto en su forma normal.
\end{teorema}

\section[Aut\'omatas de empuje]{Aut\'omatas de empuje (pushdown automata)}

La principal ventaja de esta nueva m\'aquina es el hecho de tener una pila donde almacenar s\'imbolos (no necesariamente del alfabeto de entrada) que le permite proceder respecto a lo \'ultimo que se halle en la pila.
Estas m\'aquinas tienen la misma capacidad que las gram\'aticas libres de contexto: a veces puede ser \'util definir la m\'aquina, mientras que en otros casos definir las reglas de una gram\'atica es m\'as prudente.

La dotamos de dos operaciones b\'asicas, que son empujar o \emph{pushing}, y extraer o \emph{popping.}
Tiene un funcionamiento del tipo LIFO (\emph{last in, first out}), lo que significa que lo \'ultimo que se saca es lo primero que se leer\'a.
